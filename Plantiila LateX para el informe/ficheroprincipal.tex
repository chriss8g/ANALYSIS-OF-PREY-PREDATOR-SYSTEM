\documentclass{wscpaperproc}
% \usepackage[spanish, activeacute]{babel}
\usepackage{latexsym}
%\usepackage{caption}
\usepackage{graphicx}
\usepackage{mathptmx}
\usepackage[T1]{fontenc}
\usepackage{amsmath}
\usepackage{setspace}

%
%****************************************************************************
% AUTHOR: You may want to use some of these packages. (Optional)
\usepackage{amsmath}
\usepackage{amsfonts}
\usepackage{amssymb}
\usepackage{amsbsy}
\usepackage{amsthm}
%****************************************************************************


%
%****************************************************************************
% AUTHOR: If you do not wish to use hyperlinks, then just comment
% out the hyperref usepackage commands below.

%% This version of the command is used if you use pdflatex. In this case you
%% cannot use ps or eps files for graphics, but pdf, jpeg, png etc are fine.

\usepackage[colorlinks=true,urlcolor=blue,citecolor=black,anchorcolor=black,linkcolor=red]{hyperref}
%\usepackage{hyperref}

%% The next versions of the hyperref command are used if you adopt the
%% outdated latex-dvips-ps2pdf route in generating your pdf file. In
%% this case you can use ps or eps files for graphics, but not pdf, jpeg, png etc.
%% However, the final pdf file should embed all fonts required which means that you have to use file
%% formats which can embed fonts. Please note that the final PDF file will not be generated on your computer!
%% If you are using WinEdt or PCTeX, then use the following. If you are using
%% Y&Y TeX then replace "dvips" with "dvipsone"

%%\usepackage[dvips,colorlinks=true,urlcolor=blue,citecolor=black,%
%% anchorcolor=black,linkcolor=black]{hyperref}
%****************************************************************************


%
%****************************************************************************
%*
%* AUTHOR: YOUR CALL!  Document-specific macros can come here.
%*
%****************************************************************************

% If you use theoremes
\newtheoremstyle{wsc}% hnamei
{3pt}% hSpace abovei
{3pt}% hSpace belowi
{}% hBody fonti
{}% hIndent amounti1
{\bf}% hTheorem head fontbf
{}% hPunctuation after theorem headi
{.5em}% hSpace after theorem headi2
{}% hTheorem head spec (can be left empty, meaning `normal')i

\theoremstyle{wsc}
\newtheorem{theorem}{Teorema}
\renewcommand{\thetheorem}{\arabic{theorem}}
\newtheorem{corollary}[theorem]{Corolario}
\renewcommand{\thecorollary}{\arabic{corollary}}
\newtheorem{definition}{Definición}
\renewcommand{\thedefinition}{\arabic{definition}}


%#########################################################
%*
%*  The Document.
%*
\begin{document}

%***************************************************************************
% AUTHOR: AUTHOR NAMES GO HERE
% FORMAT AUTHORS NAMES Like: Author1, Author2 and Author3 (last names)
%
%		You need to change the author listing below!
%               Please list ALL authors using last name only, separate by a comma except
%               for the last author, separate with "and"
%
\WSCpagesetup{Álvarez, Cordero, Díaz, Guerra, and Llerena}
% \pagestyle{myheadings}
% \markright{Álvarez, Cordero, Díaz, Guerra, and Llerena}

% AUTHOR: Enter the title, all letters in upper case
\title{Análisis de {\it Stability and Numerical Simulation of Prey-predator System with Holling
			Type-II Functional Responses for Adult Prey}}

% AUTHOR: Enter the authors of the article, see end of the example document for further examples
\author{
	Pedro Pablo Álvarez Portelles\\[12pt]
	Grupo C212\\
	Ciencia de la Computación\\
	Facultad de Matemática y Computación\\
	Universidad de La Habana. Cuba\\
	% Multiple authors are entered as follows.
	% You may also need to adjust the titlevbox size in the preamble - search for titlevboxsize
	\and
	Amanda Cordero Lezcano\\[12pt]
	Grupo C212\\
	Ciencia de la Computación\\
	Facultad de Matemática y Computación\\
	Universidad de La Habana. Cuba\\
	\and
	Marlon Díaz Pérez\\[12pt]
	Grupo C212\\
	Ciencia de la Computación\\
	Facultad de Matemática y Computación\\
	Universidad de La Habana. Cuba\\
	\and
	Christopher Guerra Herrero\\[12pt]
	Grupo C212\\
	Ciencia de la Computación\\
	Facultad de Matemática y Computación\\
	Universidad de La Habana. Cuba\\
	\and
	Abel Llerena Domínguez\\[12pt]
	Grupo C212\\
	Ciencia de la Computación\\
	Facultad de Matemática y Computación\\
	Universidad de La Habana. Cuba\\
}



\maketitle
\section*{Tareas a realizar}
En el Informe debe Presentar:
\begin{itemize}
	\item Informe de la Tarea Investigativa II. Título del artículo analizado
	\item Autores del trabajo.
	\item Resumen del trabajo.
	\item Intoducción del trabajo debe de mensional, los autores del artículo analizado, la revista donde se publicó. A\~no. Factor de impacto de la revista. Valoración del artículo: Explicación sobre lo que trata el artículo, problemática que se propone resolver, técnicas utilizadas.
	\item Otro epígrafe para presentar las ecuaciones que ilustran el modelo matemático utilizado. Condiciones iniciales o de frontera. Resultados a los que arriban. Ejemplos numéricos: Reproducción de los algunos de los ejemplos o experimentos numéricos que se expliquen en el artículo, utilizando para ello (RK4/Euler explícito o implícito) estudiado en clases y comparar resultados. Buscar puntos de equilibrio en caso de existir y analizar la estabilidad de dichos puntos. Pueden usarse para ello recursos computacionales. Presentar el diagrama de fases entre un par variables incógnitas, valorando su comportamiento.
	\item Conclusiones: Una valoración de lo que usted ha aprendido con este trabajo, como valora la posibilidad de que se pueda continuar esta línea de investigación.
	\item Bibliografía Consultada.
	\item  Anexos: Incluir seudo códigos de sus programas.
	\item Valoraremos las iniciativas que presenten, como pueden ser, interfaces gráficas, bases de datos, elementos  vinculen con otras asignaturas de la especialidad.
\end{itemize}

\section*{Resumen}
\textcolor{red}{Este artículo examina la estabilidad local de un modelo de presa-depredador con estructura de etapas en
	las poblaciones de presas. Se analizan los puntos de equilibrio y se realiza una exploración numérica
	para mostrar la existencia de un ciclo límite estable. Los resultados confirman los hallazgos analíticos
	y revelan el comportamiento dinámico local del modelo, considerando las interacciones entre presas
	jóvenes, adultas y depredadores. Se resalta la importancia de las condiciones para la estabilidad
	del punto de equilibrio interior. Además, se plantea la necesidad de investigar las bifurcaciones
	de Hopf en estudios futuros para comprender mejor el comportamiento global del modelo. Estos hallazgos
	contribuyen a ampliar nuestro conocimiento sobre la dinámica de los sistemas presa-depredador con
	estructura de etapas.}
\section{INTRODUCCIóN}
\label{sec:intro}
El propósito de este documento es hacer un análisis de {\it Stability and Numerical Simulation of Prey-predator System with Holling
		Type-II Functional Responses for Adult Prey}, artículo elaborado por la especialista analítica Dian Savitri. Fue presentado en el MISEIC(Matematics,
Informatics, Science and Education International Conference) en 2019 y publicado en Journal of Physics: Conference Series, bajo la licencia de IOP Publishing.
Factor de impacto de la revista: 0.227, en la fecha de publicación del artículo. Actualmente la revista tiene un factor de impacto de 0.21.


	{\bf Objetivos del artículo:} Se analiza la estabilidad local del modelo presa-depredador. El modelo fue construido a partir de dos presas que involucran
una estructura de etapa y un depredador. Posee tres equilibrios positivos: el original, la extinción del depredador y el punto interior. El artículo hace un estudio de la dinámica
del comportamiento de las interacciones presa-depredador, estructuradas por etapas con la función de respuesta Holling tipo II(esta respuesta funcional se
refiere al cambio en el comportamiento de los individuos en función de la densidad del huésped o presa) para presas adultas. En un sistema dinámico,
si hay ciclos límites, entonces el punto de equilibrio interior es el centro, en estos hace referencia a que los ciclos límites fueron generados por "Hopf Bifurcation"
(bifurcación: es el cambio de estabilidad de un sistema que se produce debido a cambios en los valores de los parámetros). El apartado considera la estabilidad de
los equilibrios en detalle con las condiciones de existencia e ilustra la estabilidad local de los equilibrios, además cuenta con simulaciones numéricas para ilustrar
los resultados.

El estudio pretende detectar los ciclos límites con sus retratos de fase y mostrar numéricamente que existe un ciclo límite estable.

\subsection*{Técnicas utilizadas:}
\begin{itemize}
	\item Criterio Routh Hurwitz: consiste en un simple procedimiento o algoritmo para poder determinar si existe alguna raíz o polo en el semiplano derecho del
	      plano complejo “s”, donde si al menos existe una raíz el sistema es inestable, caso contrario si no hay ninguna raíz en el semiplano derecho el sistema es estable.
	\item Usando python y una implementación computacional del método Runge-Kutta de orden 4 se resolvió el sistema presa-depredador.
\end{itemize}
\subsection{Estructura del trabajo}
*Insertar especie d índice

\section{Resultados fundamentales.}


Ecuaciones que ilustran el modelo matemático utilizado:

\begin{equation}\label{dx}
	\frac{dx}{dt} = rx(1-\frac{x}{k})-\beta x-\alpha xz
\end{equation}
\begin{equation}\label{dy}
	\frac{dy}{dt} = \beta x-\frac{\eta yz}{y+m}-\mu y
\end{equation}
\begin{equation}\label{dz}
	\frac{dz}{dt} = \alpha_1 xz+\rho z^2-\frac{\eta_1z^2}{y+m}
\end{equation}

\vspace*{1cm}

El término $rx(1-\frac{x}{k})$ de (\ref*{dx}) es conocido como ecuación
logística y se compone por:
\vspace*{0.3cm}

$x$: población de presas juveniles.

$t$: tiempo.

$r$ constante que define la tasa de crecimiento.

$k$: capacidad de carga o persistencia.

\vspace*{0.5cm}

Esta primera ecuación representa el modelo de crecimiento poblacional en las
presas jóvenes. Para ello se tuvo en cuenta:
\begin{itemize}
	\item La tasa de reproducciónes proporcional a la población existente.
	\item La tasa de reproducción es proporcional a la cantidad de recursos disponibles.
	\item La competición por los recursos disponibles tiende a limitar el crecimiento poblacional.
\end{itemize}

$\beta x$ representa las presas que se vuelven adultas.

\vspace*{0.3cm}

$\beta$: factor de conversión de presa joven a presa adulta

\vspace*{0.3cm}

La velocidad con que varía la población de presas $x$ es proporcional al número de encuentros con los depredadores $z$ según la ecuación de Lotka-Volterra:
$$\frac{dx}{dt}=\alpha xz$$
$\alpha$: tasa de eliminación de presas por parte de los depredadores.

\vspace*{1cm}

Por otra parte (\ref*{dy}) se refiere a la población de presas adultas.

El primer término de la ecuación ya fue analizado anteriormente, es la cantidad de presas jóvenes que se convierten en adultas.

El proceso de eliminación de presas por los depredadores se describe a partir del modelo Holling Tipo II, que establece:

$$\frac{dy}{dt} = -\frac{\eta yz}{y+m}y$$

donde:

$\eta$: valor máximo de la tasa de reducción per cápita de presas adultas debido a los depredadores.

$m$: coeficiente de protección ambiental para las presas adultas.

Además se considera las presas que mueren de forma natural $\mu y$.

$\mu$: tasa de mortalidad en presas adultas.


\vspace*{1cm}

Por último (\ref*{dz}) modela la población de depredadores. $\alpha_1 xz+\rho z^2$ representa el crecimiento según disponibilidad de alimentos favoritos de los depredadores.

$\alpha_1$: tasa de disponibilidad de alimentos favoritos de los depredadores.

\vspace*{0.3cm}

$\rho z^2$ es el crecimiento intrínseco.

$\rho$: tasa de crecimiento intrínseco.

\vspace*{0.3cm}

Además se considera que esta población decrece a partir del modelo de Leslie-Gower modificado:

$$\frac{dz}{dt} =-\frac{\rho z^2}{n y+c}$$

$$\frac{dz}{dt} =-\frac{\frac{\rho}{x} z^2}{y+\frac{c}{n}}$$

$n$: calidad energética de la presa como alimento.

$c$: tamaño máximo de disponibilidad de alimentos alternativos.

\vspace*{0.5cm}

$$\frac{\rho}{n}=n_1$$

$$\frac{c}{n}=m$$

$n_1$: crecimiento intríseco dividido por el aporte de las presas al depredador.

$m$: impacto de la escacez de presas, en la supervivencia y reproducción de los depredadores.


\vspace*{3cm}

\section*{Anális de los puntos de equilibrio y su estabilidad}

Los puntos de equilibrio para este sistema son aquellos que satisfacen:
$$\frac{dx}{dt}=\frac{dy}{dt}=\frac{dz}{dt}=0$$

Estos puntos son:
\begin{itemize}
	\item $P_1=(0, 0, 0)$
	\item $P_2=(\frac{k(r-\beta)}{r}, \frac{\beta k(r-\beta)}{\mu r}, 0)$
	\item $P_3=(x_3, y_3, z_3)$
\end{itemize}

La estabilidad de los puntos de equilibrio se puede analizar calculanfo la matriz Jacobiana. Para nuestro sistema sería:
$$ J(x^*, y^*, z^*) = \left(
	\begin{array}{ccc}
			r-\frac{2rx}{k}-\beta-\alpha z & 0                                       & -\alpha x                       \\
			\beta                          & \frac{-nz}{y+m}+\frac{nyz}{(y+m)^2}-\mu & \frac{-ny}{y+m}                 \\
			\alpha_1z                      & \frac{n_1z^2}{(y+m)^2}                  & 2pz-\frac{2n_1z}{y+m}+\alpha_1x
		\end{array}
	\right)$$

La ecuación característica esta dada pot el $det(J(x^*, y^*, z^*)-I)=0$ donde $(x^*, y^*, z^*)$ es un punto de equilibrio del sistema.
Nosotros solo analizaremos la estabilidad de los puntos de equilibrio no negativos por la parte real de los eigenvalores de la matriz Jacobiana.
\vspace*{0.5cm}
Para $P_1$ sería:
$$\left|
	\begin{array}{ccc}
		(-\beta+r)-\lambda & 0            & 0        \\
		\beta              & -\mu-\lambda & 0        \\
		0                  & 0            & -\lambda
	\end{array}
	\right| =0$$
De ahí que sus eigenvalores están dados por $((-\beta+r)-\lambda)(-\mu-\lambda)(-\lambda)=0$. Pero esta ecuación siempre tiene valor negativo
\textcolor{red}{$\lambda_1=-$}, también si $r<\beta$, pero \textcolor{red}{$\lambda_3=0$}. Por tanto en $P_1$ el sistema es inestable.

\vspace*{0.5cm}

Para $P_2$ sería:
$$\left|
	\begin{array}{ccc}
		(\beta+r)-\lambda & 0            & \frac{\alpha k(\beta-r)}{r}                                      \\
		\beta             & -\mu-\lambda & \frac{n\beta k(\beta-r)}{r\mu (\frac{\beta k(r-\beta)}{r\mu}+m)} \\
		0                 & 0            & \frac{\alpha_1 k(\beta-r)}{r}-\lambda
	\end{array}
	\right| =0$$

Que luego de un poco de trabajo algebraico obtendrías $\textcolor{red}{\lambda_1=\beta-r, \lambda_2=cosa q no entiendo, \lambda_3=\frac{\alpha_1 k(\beta-r)}{r}}$. Por lo que $P_2$ es inestable.

\vspace*{0.5cm}

En el caso $P_3$ sería:
$$ J(x_3, y_3, z_3) = \left(
	\begin{array}{ccc}
			r-\frac{2rx_3}{k}-\beta-\alpha z_3 & 0                                                 & -\alpha x_3                             \\
			\beta                              & \frac{-nz_3}{y_3+m}+\frac{ny_3z_3}{(y_3+m)^2}-\mu & \frac{-ny_3}{y_3+m}                     \\
			\alpha_1z_3                        & \frac{n_1z_3^2}{(y_3+m)^2}                        & 2pz_3-\frac{2n_1z_3}{y_3+m}+\alpha_1x_3
		\end{array}
	\right)$$

La ecuación característica entonces es de la forma:
$$\lambda^3+\gamma_1\lambda^2+\gamma_2\lambda+\gamma_3=0 $$

Si usamos el criterio Routh-Hurwitz para analizar la estabilidad en $P_3$ veremos que este
tiene una parte real negativa si y solo si $\gamma_1>0, \gamma_3>0$ y $\gamma_1\gamma_2-\gamma_3>0$, la coexistencia en $P_3$ es local
y asintóticamente estable.

\vspace*{3cm}


\section*{Métodos y algoritmos utilizados}


\subsection*{Ejemplos Numéricos. Reproducción de los experimentos}

Para reproducir los experimentos se utilizó el método de Runge-Kutta de orden 4 (RK4), programado
en Python 3.10 y haciendo uso de librerías para la manipulación eficiente de vectores como Numpy y
Scipy.

El metodo de Runge-Kutta es método de un paso en el cual se usa un tamaño de paso $h$ para calcular
los siguientes valores a partir de valores previos.

Como en este trabajo se utiliza un sistema de tres ecuaciones,
es necesario de manera análoga crear otras variables $l$ y $m$ para
las funciones $y$ y $z$. El pseudocódigo sería de la siguiente manera:


\doublespacing
\begin{equation*}
	\begin{split}
		for\ i=0\ to\ n - 1:\\
		&xn = x[i];yn = y[i];zn = z[i]\\
		\\
		&k_1 = f_1(t_n, x_n, y_n, z_n)\\
		&l_1 = f_2(t_n, x_n, y_n, z_n)\\
		&m_1 = f_3(t_n, x_n, y_n, z_n)\\
		&k_2 = f_1(t_n + \frac{h}{2}, x_n + \frac{k_1}{2}, y_n + \frac{l_1}{2}, z_n + \frac{m_1}{2})\\
		&l_2 = f_2(t_n + \frac{h}{2}, x_n + \frac{k_1}{2}, y_n + \frac{l_1}{2}, z_n + \frac{m_1}{2})\\
		&m_2 = f_3(t_n + \frac{h}{2}, x_n + \frac{k_1}{2}, y_n + \frac{l_1}{2}, z_n + \frac{m_1}{2})\\
		&k_3 = f_1(t_n + \frac{h}{2}, x_n + \frac{k_2}{2}, y_n + \frac{l_2}{2}, z_n + \frac{m_2}{2})\\
		&l_3 = f_2(t_n + \frac{h}{2}, x_n + \frac{k_2}{2}, y_n + \frac{l_2}{2}, z_n + \frac{m_2}{2})\\
		&m_3 = f_3(t_n + \frac{h}{2}, x_n + \frac{k_2}{2}, y_n + \frac{l_2}{2}, z_n + \frac{m_2}{2})\\
		&k_4 = f_1(t_n + h, x_n + k_3, y_n + l_3, z_n + m_3)\\
		&l_4 = f_2(t_n + h, x_n + k_3, y_n + l_3, z_n + m_3)\\
		&m_4 = f_3(t_n + h, x_n + k_3, y_n + l_3, z_n + m_3)\\
		\\
		&t[i+1] = t[i] + h\\
		&x[i+1] = \frac{h}{6}(k_1 + \frac{k_2}{2} + \frac{k_3}{3} + k_4)\\
		&y[i+1] = \frac{h}{6}(l_1 + \frac{l_2}{2} + \frac{l_3}{3} + l_4)\\
		&z[i+1] = \frac{h}{6}(m_1 + \frac{m_2}{2} + \frac{m_3}{3} + m_4)\\
	\end{split}
\end{equation*}
\singlespacing

El tamaño de paso se estableció para todos los experimentos en
$h=2^{-10}$ con el cual se obtienen resultados muy exactos teniendo
en cuenta que el error de RK4 es del orden $O(h^5)$, con lo que tendriamos
que el error sería aproximadamente del orden de $e \approx 2^{-50}$.

\subsection*{Reproducción de los experimentos}

Para los experimentos se utilizaron para los parámetros los valores de la siguiente tabla:

\begin{tabular}{p{1cm} | p{1cm} | p{1cm} | p{1cm} | p{1cm} | p{1cm} | p{1cm} | p{1cm} | p{1cm} }
	% aslask & gahbh
	$\#$ & $r$  & $\alpha$ & $\alpha_1$ & $\eta$ & $\eta_1$ & $K$  & $\rho$ & $m$  \\
	1    & 0.82 & 1.56     & 1.12       & 2.41   & 1.83     & 12.0 & 1.38   & 0.13 \\
	2    & 1.32 & 1.56     & 0.72       & 2.41   & 0.41     & 2.8  & 1.38   & 0.23 \\
	3    & 1.32 & 0.76     & 0.72       & 0.6    & 0.41     & 2.8  & 0.78   & 0.23 \\
	4    & 0.82 & 0.76     & 0.72       & 1.2    & 0.41     & 2.8  & 1.38   & 0.23 \\
	5    & 1.32 & 1.16     & 0.72       & 0.31   & 0.41     & 2.8  & 0.78   & 0.23 \\
\end{tabular}
\\

Además, se usó fijo $\beta=0.87$ y $\mu=0.11$.

\vspace*{0.5cm}
{\bf Primer experimento}

En el primer conjunto de experimentos se usaron los valores iniciales $[x_0=3.01, y_0=5.05, z_0=4.28]$ (1.1)
y $[x_0=4.6, y_0=5.9, z_0=3.1]$(1.2).

\begin{figure}[h!]
	\includegraphics[width=\linewidth]{./images/1.png}
	\caption{Experimento \#1}
\end{figure}

\vspace*{0.5cm}
{\bf Segundo experimento}

En el segundo conjunto de experimentos, se usaron valores tal que $\eta > \beta$ y $\eta > \alpha$.
De valores iniciales se utilizaron $[x_0=0.3, y_0=2.4, z_0=3.9]$(2.1),
$[x_0=0.6, y_0=2.4, z_0=3.9]$(2.2) y $[x_0=2.1, y_0=1.2, z_0=1.1]$

La simulación muestra que todos los valores van al punto de equilibrio interior.

\begin{figure}[h!]
	\includegraphics[width=\linewidth]{./images/2.png}
	\caption{Experimento \#2}
\end{figure}

\vspace*{1cm}
{\bf Tercer experimento}

En el tercer experimento, se usaron valores tal que $\alpha > \beta$
Los valores iniciales usados fueron $[x_0=0.3, y_0=2.4, z_0=3.9]$. La simulación muestra
que los valores van al punto de equilibrio.

\begin{figure}[h!]
	\includegraphics[width=\linewidth]{./images/3.png}
	\caption{Experimento \#3}
\end{figure}

\vspace*{0.5cm}
{\bf Cuarto experimento}

En el cuarto experimento, se usaron valores tal que $\eta = \alpha$.
Los valores iniciales usados fueron $[x_0=1.2, y_0=2.1, z_0=4.28]$. La simulación muestra
que los valores van al punto de equilibrio.

\begin{figure}[h!]
	\includegraphics[width=\linewidth]{./images/4.png}
	\caption{Experimento \#4}
\end{figure}

\vspace*{7cm}
{\bf Quinto experimento}

En el quinto experimento, se usaron valores tal que $\eta > \beta$ y $\eta > \alpha$.
Los valores iniciales usados fueron $[x_0=1.2, y_0=2.1, z_0=2.4]$. La simulación muestra
que los valores van al punto de equilibrio.

\begin{figure}[!h]
	\includegraphics[width=\linewidth]{./images/5.png}
	\caption{Experimento \#5}
\end{figure}
\vspace*{1cm}


\section*{Conclusiones}

\textcolor{red}{
	Este estudio ha examinado la estabilidad local de un modelo de interacción presa-depredador con estructura
	de etapas en las poblaciones de presas. Se han identificado y analizado los puntos de equilibrio,
	encontrando tres equilibrios positivos: el original, la extinción del depredador y el punto interior.
	Se ha demostrado que el punto interior es localmente estable bajo ciertas condiciones. Las simulaciones
	numéricas han respaldado los resultados analíticos, proporcionando evidencia adicional del comportamiento
	local y global del modelo. Además, se ha detectado la existencia de un ciclo límite estable, lo que destaca
	la importancia de considerar la estructura de etapas en la dinámica presa-depredador. Estos hallazgos
	contribuyen a mejorar nuestra comprensión de los sistemas ecológicos y proporcionan una base sólida para
	futuras investigaciones sobre bifurcaciones y estabilidad global en este tipo de modelos.
}

% {\footnotesize
% 	\begin{hangref}
% 		\item \textcolor{red}{Banks, J., J. S. Carson, B. L. Nelson, and D. M. Nicol. 2000. \textit{Discrete-Event System Simulation}. 3rd ed. Upper Saddle River, New Jersey: Prentice-Hall, Inc.}
% 	\end{hangref}
% }




% \bibliographystyle{wsc}
% \bibliography{demobib}




\section*{Agradciemientos}

-Creo q nadie :)
% \ppendix

\section{Anexos}
-Tercero B?


\end{document}

