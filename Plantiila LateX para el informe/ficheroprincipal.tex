\documentclass{wscpaperproc}
\usepackage[spanish, activeacute]{babel}
\usepackage{latexsym}
%\usepackage{caption}
\usepackage{graphicx}
\usepackage{mathptmx}
\usepackage[T1]{fontenc}

%
%****************************************************************************
% AUTHOR: You may want to use some of these packages. (Optional)
\usepackage{amsmath}
\usepackage{amsfonts}
\usepackage{amssymb}
\usepackage{amsbsy}
\usepackage{amsthm}
%****************************************************************************


%
%****************************************************************************
% AUTHOR: If you do not wish to use hyperlinks, then just comment
% out the hyperref usepackage commands below.

%% This version of the command is used if you use pdflatex. In this case you
%% cannot use ps or eps files for graphics, but pdf, jpeg, png etc are fine.

\usepackage[colorlinks=true,urlcolor=blue,citecolor=black,anchorcolor=black,linkcolor=red]{hyperref}
%\usepackage{hyperref}

%% The next versions of the hyperref command are used if you adopt the
%% outdated latex-dvips-ps2pdf route in generating your pdf file. In
%% this case you can use ps or eps files for graphics, but not pdf, jpeg, png etc.
%% However, the final pdf file should embed all fonts required which means that you have to use file
%% formats which can embed fonts. Please note that the final PDF file will not be generated on your computer!
%% If you are using WinEdt or PCTeX, then use the following. If you are using
%% Y&Y TeX then replace "dvips" with "dvipsone"

%%\usepackage[dvips,colorlinks=true,urlcolor=blue,citecolor=black,%
%% anchorcolor=black,linkcolor=black]{hyperref}
%****************************************************************************


%
%****************************************************************************
%*
%* AUTHOR: YOUR CALL!  Document-specific macros can come here.
%*
%****************************************************************************

% If you use theoremes
\newtheoremstyle{wsc}% hnamei
{3pt}% hSpace abovei
{3pt}% hSpace belowi
{}% hBody fonti
{}% hIndent amounti1
{\bf}% hTheorem head fontbf
{}% hPunctuation after theorem headi
{.5em}% hSpace after theorem headi2
{}% hTheorem head spec (can be left empty, meaning `normal')i

\theoremstyle{wsc}
\newtheorem{theorem}{Teorema}
\renewcommand{\thetheorem}{\arabic{theorem}}
\newtheorem{corollary}[theorem]{Corolario}
\renewcommand{\thecorollary}{\arabic{corollary}}
\newtheorem{definition}{Definición}
\renewcommand{\thedefinition}{\arabic{definition}}


%#########################################################
%*
%*  The Document.
%*
\begin{document}

%***************************************************************************
% AUTHOR: AUTHOR NAMES GO HERE
% FORMAT AUTHORS NAMES Like: Author1, Author2 and Author3 (last names)
%
%		You need to change the author listing below!
%               Please list ALL authors using last name only, separate by a comma except
%               for the last author, separate with "and"
%
\WSCpagesetup{Álvarez, Cordero, Díaz, Guerra, and Llerena}

% AUTHOR: Enter the title, all letters in upper case
\title{Análisis de {\it Stability and Numerical Simulation of Prey-predator System with Holling
			Type-II Functional Responses for Adult Prey}}

% AUTHOR: Enter the authors of the article, see end of the example document for further examples
\author{
	Pedro Pablo Álvarez Portelles\\[12pt]
	Grupo C212\\
	Ciencia de la Computación\\
	Facultad de Matemática y Computación\\
	Universidad de La Habana. Cuba\\
	% Multiple authors are entered as follows.
	% You may also need to adjust the titlevbox size in the preamble - search for titlevboxsize
	\and
	Amanda Cordero Lezcano\\[12pt]
	Grupo C212\\
	Ciencia de la Computación\\
	Facultad de Matemática y Computación\\
	Universidad de La Habana. Cuba\\
	\and
	Marlon Díaz Pérez\\[12pt]
	Grupo C212\\
	Ciencia de la Computación\\
	Facultad de Matemática y Computación\\
	Universidad de La Habana. Cuba\\
	\and
	Christopher Guerra Herrero\\[12pt]
	Grupo C212\\
	Ciencia de la Computación\\
	Facultad de Matemática y Computación\\
	Universidad de La Habana. Cuba\\
	\and
	Abel Llerena Domínguez\\[12pt]
	Grupo C212\\
	Ciencia de la Computación\\
	Facultad de Matemática y Computación\\
	Universidad de La Habana. Cuba\\
}



\maketitle
\section*{Tareas a realizar}
En el Informe debe Presentar:
\begin{itemize}
	\item Informe de la Tarea Investigativa II. Título del artículo analizado
	\item Autores del trabajo.
	\item Resumen del trabajo.
	\item Intoducción del trabajo debe de mensional, los autores del artículo analizado, la revista donde se publicó. A\~no. Factor de impacto de la revista. Valoración del artículo: Explicación sobre lo que trata el artículo, problemática que se propone resolver, técnicas utilizadas.
	\item Otro epígrafe para presentar las ecuaciones que ilustran el modelo matemático utilizado. Condiciones iniciales o de frontera. Resultados a los que arriban. Ejemplos numéricos: Reproducción de los algunos de los ejemplos o experimentos numéricos que se expliquen en el artículo, utilizando para ello (RK4/Euler explícito o implícito) estudiado en clases y comparar resultados. Buscar puntos de equilibrio en caso de existir y analizar la estabilidad de dichos puntos. Pueden usarse para ello recursos computacionales. Presentar el diagrama de fases entre un par variables incógnitas, valorando su comportamiento.
	\item Conclusiones: Una valoración de lo que usted ha aprendido con este trabajo, como valora la posibilidad de que se pueda continuar esta línea de investigación.
	\item Bibliografía Consultada.
	\item  Anexos: Incluir seudo códigos de sus programas.
	\item Valoraremos las iniciativas que presenten, como pueden ser, interfaces gráficas, bases de datos, elementos  vinculen con otras asignaturas de la especialidad.
\end{itemize}

\section*{Estructura de la platilla}
\section*{Resumen}
Este artículo examina la estabilidad local de un modelo de presa-depredador con estructura de etapas en
las poblaciones de presas. Se analizan los puntos de equilibrio y se realiza una exploración numérica
para mostrar la existencia de un ciclo límite estable. Los resultados confirman los hallazgos analíticos
y revelan el comportamiento dinámico local del modelo, considerando las interacciones entre presas
jóvenes, adultas y depredadores. Se resalta la importancia de las condiciones para la estabilidad
del punto de equilibrio interior. Además, se plantea la necesidad de investigar las bifurcaciones
de Hopf en estudios futuros para comprender mejor el comportamiento global del modelo. Estos hallazgos
contribuyen a ampliar nuestro conocimiento sobre la dinámica de los sistemas presa-depredador con
estructura de etapas.
\section{INTRODUCCIóN}
\label{sec:intro}
El propósito de este documento es hacer un análisis de {\it Stability and Numerical Simulation of Prey-predator System with Holling
		Type-II Functional Responses for Adult Prey}, artículo elaborado por ña especialista analítica Dian Savitri.Fue presentado en el MISEIC(Matematics,
Informatics, Science and Education International Conference) en 2019 y publicado en Journal of Physics: Conference Series, bajo la licencia de IOP Publishing.
Factor de impacto de la revista: 0.227 (en la fecha de publicación del artículo, actualmente la revista tiene un factor de impacto de 0.21).

	{\bf Objetivos del artículo:} Se analiza la estabilidad local del modelo presa-depredador. El modelo fue construido a partir de dos presas que involucran
una estructura de etapa y un depredador, y tiene una manera mucho más simple de simular la diversidad que los modelos existentes, y presenta nuevos fenómenos
en el mundo real. Posee tres equilibrios positivos: el original, la extinción del depredador y el punto interior. El artículo hace un estudio de la dinámica
del comportamiento de las interacciones presa-depredador, estructuradas por etapas con la función de respuesta Holling tipo II(esta respuesta funcional se
refiere al cambio en el comportamiento de los individuos en función de la densidad del huésped o presa) para presas adultas. En un sistema dinámico,
si hay ciclos límites, entonces el punto de equilibrio interior es el centro, en estos hace referencia a que los ciclos límites fueron generados por "Hopf Bifurcation"
(bifurcación: es el cambio de estabilidad de un sistema que se produce debido a cambios en los valores de los parámetros). El apartado considera la estabilidad de
los equilibrios en detalle con las condiciones de existencia e ilustra la estabilidad local de los equilibrios, además cuenta con simulaciones numéricas para ilustrar
los resultados.

El estudio pretende detectar los ciclos límites con sus retratos de fase y mostrar numéricamente que existe un ciclo límite estable.
Técnicas utilizadas:
\begin{itemize}
	\item Criterio Routh Hurwitz: consiste en un simple procedimiento o algoritmo para poder determinar si existe alguna raíz o polo en el semiplano derecho del
	      plano complejo “s”, donde si al menos existe una raíz el sistema es inestable, caso contrario si no hay ninguna raíz en el semiplano derecho el sistema es estable.
	\item Usando python y una implementación computacional del método Runge-Kutta de orden 4 se resolvió el sistema presa-depredador.
	\item Respuesta funcional de Beddington De-Angelis: similar a Holling tipo II, pero contiene un término adicional que describe la interferencia mutua de los depredadores.
\end{itemize}
\subsection{Estructura del trabajo}
*Insertar especie d índice

\section{Resultados fundamentales.}

Muestre sólo las ecuaciones más importantes y numere únicamente las ecuaciones mostradas a las que se hace referencia explícita en el texto. \\

$$s^2 = \frac 1 {n-1} \sum_{i=1}^n (Y_i - \bar Y)^2.$$

\begin{equation}%\label{eq:quadratic2}
	ax^2 + bx + c = 0, \mbox{ donde } a \ne 0.
\end{equation}

En el texto, cada referencia a un número de ecuación debe ir también entre paréntesis. Por ejemplo, la solución de está dada por  en los Axenos .


\begin{equation}% \label{eq:quadratic_second}
	ax^2 + bx + c = 0
\end{equation}

\subsection{Métodos y algoritmos utilizados}
Técnicas utilizadas:
\begin{itemize}
	\item Criterio Routh Hurwitz: consiste en un simple procedimiento o algoritmo para poder determinar si existe alguna raíz o polo en el semiplano derecho del
	      plano complejo “s”, donde si al menos existe una raíz el sistema es inestable, caso contrario si no hay ninguna raíz en el semiplano derecho el sistema es estable.
	\item Usando python y una implementación computacional del método Runge-Kutta de orden 4 se resolvió el sistema presa-depredador.
	\item Respuesta funcional de Beddington De-Angelis: similar a Holling tipo II, pero contiene un término adicional que describe la interferencia mutua de los depredadores.
\end{itemize}

Este estudio ha examinado la estabilidad local de un modelo de interacción presa-depredador con estructura
de etapas en las poblaciones de presas. Se han identificado y analizado los puntos de equilibrio,
encontrando tres equilibrios positivos: el original, la extinción del depredador y el punto interior.
Se ha demostrado que el punto interior es localmente estable bajo ciertas condiciones. Las simulaciones
numéricas han respaldado los resultados analíticos, proporcionando evidencia adicional del comportamiento
local y global del modelo. Además, se ha detectado la existencia de un ciclo límite estable, lo que destaca
la importancia de considerar la estructura de etapas en la dinámica presa-depredador. Estos hallazgos
contribuyen a mejorar nuestra comprensión de los sistemas ecológicos y proporcionan una base sólida para
futuras investigaciones sobre bifurcaciones y estabilidad global en este tipo de modelos.


	{\footnotesize
		\begin{hangref}
			\item Banks, J., J. S. Carson, B. L. Nelson, and D. M. Nicol. 2000. \textit{Discrete-Event System Simulation}. 3rd ed. Upper Saddle River, New Jersey: Prentice-Hall, Inc.
		\end{hangref}
	}




% \bibliographystyle{wsc}
% \bibliography{demobib}




\section*{Agradciemientos}


% \appendix

\section{Anexos}


\end{document}

